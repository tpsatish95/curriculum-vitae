%%%%%%%%%%%%%%%%%%%%%%%%%%%%%%%%%%%%%%%%%%%%%%%%%%%%%%%%%%%%%%%%%%%%%
% Author: Maria Luiza Linhares Dantas
%
% This is an example of a very complete CV using the 'moderncv' package
% and the 'timeline' package. For more information on those, please
% access:
% https://www.ctan.org/tex-archive/macros/latex/contrib/moderntimeline
% https://www.ctan.org/tex-archive/macros/latex/contrib/moderncv
%%%%%%%%%%%%%%%%%%%%%%%%%%%%%%%%%%%%%%%%%%%%%%%%%%%%%%%%%%%%%%%%%%%%%
%\title{Curriculum Vitae}
\documentclass[11pt, a4paper, roman]{moderncv}
\moderncvstyle{classic}
\moderncvcolor{blue}
\pagenumbering{roman}
\usepackage{tabularx}
\usepackage[T1]{fontenc}
\usepackage[utf8]{inputenc}
\usepackage[scale=0.85]{geometry}    % Width of the entire CV
\setlength{\hintscolumnwidth}{2.2cm} % Width of the timeline on your left
\usepackage{mathptmx}
\usepackage{pdfpages/pdfpages}
\usepackage{moderntimeline/moderntimeline}
\usepackage{xpatch/xpatch}
\usepackage{color, graphicx}
\tlmaxdates{2007}{2016}              % Beggining and start of your timeline

\usepackage[none]{hyphenat}

\usepackage{lastpage}
%\lhead{\addressfont\itshape{UW-Madison Department of Computer Sciences}}

%\rhead{\addressfont\itshape{\textbf{U-M ID: 51481037}}}
\rfoot{\addressfont\itshape{Page \thepage\ }}

\usepackage{array}
\newcolumntype{P}[1]{>{\centering\arraybackslash}p{#1}}


\newcommand{\tab}[1]{\hspace{.2667\textwidth}\rlap{#1}}
\newcommand{\itab}[1]{\hspace{0em}\rlap{#1}}

\newcommand{\cvreferencecolumn}[2]{%
  \cvitem[0.8em]{}{%
    \begin{minipage}[t]{\listdoubleitemmaincolumnwidth}#1\end{minipage}%
    \hfill%
    \begin{minipage}[t]{\listdoubleitemmaincolumnwidth}#2\end{minipage}%
    }%
}

\newcommand{\cvreference}[8]{%
    \textbf{#1}\newline% Name
    \ifthenelse{\equal{#2}{}}{}{\addresssymbol~#2\newline}%
    \ifthenelse{\equal{#3}{}}{}{#3\newline}%
    \ifthenelse{\equal{#4}{}}{}{#4\newline}%
    \ifthenelse{\equal{#5}{}}{}{#5\newline}%
    \ifthenelse{\equal{#6}{}}{}{\emailsymbol~\texttt{\href{mailto:#6}{\nolinkurl{#6}}}\newline}%
    \ifthenelse{\equal{#7}{}}{}{\phonesymbol~#7\newline}
    \ifthenelse{\equal{#8}{}}{}{\mobilephonesymbol~#8}}


% Personal Information
\name{Satish Palaniappan}{}
\title{\emph{Curriculum Vitae}}
\social[github]{tpsatish95}   % optional, remove / comment the line if not wanted
\social[linkedin]{satishpalaniappan}  % optional, remove / comment the line if not wanted
\email{tpsatish95@gmail.com}                % optional, remove / comment the line if not wanted
\phone[mobile]{+91~9488515784}

\begin{document}
\pagenumbering{arabic}
\makecvtitle

\section{Interests}
\cventry{}{Machine Learning, Deep Learning, Reinforcement Learning, Computer Vision, Natural Language Processing, Algorithm Design}{}{}{}{}

%%%%%%%%%%%%%%%%%%%%%%%%%%%%%%%%%%%%%%%%%%%%%%%%%%%%%%%%%%%%%%%%%%%%%%%%%

\section{Educational Background}
\cventry{2012-2016}{Sri Sivasubramaniya Nadar College of Engineering (SSN CE)}{Anna University, \newline B.E. Computer Science and Engineering, \emph{\textbf{CGPA}: 8.56/10}}{}{}{}
\cventry{2010-2012}{TVS Matriculation Higher Secondary School (TVS M HSS)}{\newline 12th Grade (Higher Secondary Certificate), \emph{\textbf{Score}: 98\%}}{}{}{}

%%%%%%%%%%%%%%%%%%%%%%%%%%%%%%%%%%%%%%%%%%%%%%%%%%%%%%%%%%%%%%%%%%%%%%%%%%%%

\section{Academic Research Experience}
\cventry{Dec,2015 - Present}{Research Assistant}{\textsc{Institute of Mathematical Sciences, Chennai, India}}{}{}{
\textbf{Optical Character Recognition on Indus Scripts} \href{https://github.com/tpsatish95/OCR-on-Indus-Seals}{{\color{blue!70}{[link]}}}
\newline
under Prof. Ronojoy Adhikari, Department of Physics
\begin{itemize}
\item To recognize Indus script symbols from scans and photographs of ancient Harappan civilization artifacts.
\item Techniques: Convolutional Neural Networks (CNN), Recurrent Neural Networks (RNN), Region based CNN (RCNN), Selective Search.
\item This work got featured in \href{http://www.thehindu.com/sci-tech/science/chennai-team-taps-ai-to-read-indus-script/article17448690.ece}{{\color{blue!70}{\textit{The Hindu}}}}, \href{http://timesofindia.indiatimes.com/city/chennai/app-may-help-decipher-indus-valley-symbols/articleshow/57281369.cms}{{\color{blue!70}{\textit{Times of India}}}}, \href{http://www.theverge.com/2017/1/25/14371450/indus-valley-civilization-ancient-seals-symbols-language-algorithms-ai}{{\color{blue!70}{\textit{The Verge}}}} (small piece), and \href{http://www.sbs.com.au/yourlanguage/tamil/en/content/app-decipher-ancient-symbols?language=en}{{\color{blue!70}{\textit{SBS Radio - Australia}}}}.
\end{itemize}}

\cventry{Dec,2014}{Research Intern}{IPTSE Winter School, \textsc{Carnegie Mellon University}}{at NIT Surathkal}{}{
\textbf{Text Based Emotion Recognition System} \href{https://github.com/tpsatish95/emotion-detection-from-text}{\color{blue!70}{[link]}}
\newline
under Prof. Bhiksha Raj, Department of Computer Science
\begin{itemize}
\item Classify the text data using histograms of word2vec word clusters into the seven basic emotions.
\item Techniques: Artificial Neural Networks (ANN), Word2Vec, Latent Dirichlet Allocation (LDA), K-means, Support Vector Machines (SVM), Plutchik's wheel of emotions, Probabilistic N-Gram models, Multinomial Naive Bayes.
\end{itemize}}

%%%%%%%%%%%%%%%%%%%%%%%%%%%%%%%%%%%%%%%%%%%%%%%%%%%%%%%%%%%%%%%%%%%%%%%%%%%%%%%

\section{Industry Research Experience}
\cventry{Jun,2016 - Present}{Associate Software Engineer}{\textsc{Qube Cinema Technologies, Chennai, India}}{}{}{
\begin{itemize}
\item Projects:
\begin{itemize}
\item Real-time adaptive partner \textbf{selection algorithm} for businesses.
\item Scalable \textbf{deep learning based computer vision algorithm} for mining the viewer demographics such as count, age, gender and emotions from periodic snaps of the theater auditorium.
\item \textbf{Video super resolution} of an entire movie by up to four times, with minimal loss in clarity.
\end{itemize}
\item Consulted by: \emph{The Chennai Mathematical Institute} (CMI).
\item Techniques and Tools: Generative Adversarial Networks (GAN), CNN for Regression, Knapsack Problem, Camera Calibration, Flask API, uWSGI+Nginx servers and Amazon Web Services - EC2, S3, SQS.
\end{itemize}}

\cventry{May,2015 - Jul,2015}{Data Scientist - Intern}{\textsc{Serendio Inc., Chennai, India}}{}{}{
\begin{itemize}
\item Projects:
\begin{itemize}
\item Universal multi-domain \textbf{sentiment scorer} for text  \href{https://github.com/tpsatish95/Universal-MultiDomain-Sentiment-Classifier}{\color{blue!70}{[link]}}.
\item \textbf{Topic composition modeling} using hierarchical K-Means and semantic word clusters \href{https://github.com/tpsatish95/Topic-Modeling-Social-Network-Text-Data}{\color{blue!70}{[link]}}.
\item \textbf{Internet slang text parser} \href{https://github.com/tpsatish95/SocialTextFilter}{\color{blue!70}{[link]}}.
\end{itemize}
\item Techniques and Tools:  Tf-idf features, bagging and boosting, Gensim, CMU ARK's Twokenize, Rake - Keyword extractor, RESTful APIs, Web crawlers.
\item Open Source Community Manager - Intern, at Serendio Labs, the open source division of Serendio Inc.
\item Serendio's campus ambassador at SSN CE.
\end{itemize}
}

%%%%%%%%%%%%%%%%%%%%%%%%%%%%%%%%%%%%%%%%%%%%%%%%%%%%%%%%%%%%%%%%%%%%%%%%%%%%%%%%

\section{Projects}
\cventry{Jun,2015 - Apr,2016}{Automated Scenario Description for Images}{Undergraduate thesis}{\textsc{Anna University}}{}{under Prof. Milton R.S., Department of Computer Science, SSN CE, \href{https://github.com/tpsatish95/image-captioning}{\color{blue!70}{[link]}}.
\begin{itemize}
\item Harnessed object identification and scene classification techniques to automatically generate natural language descriptions of images.
\end{itemize}}

\cventry{Feb,2015 - Dec,2015}{Automated Scoring of YouTube videos from Pairwise Comparisons and Metadata based on Degree of Funniness}{Side project}{\href{https://github.com/tpsatish95/Youtube-Comedy-Comparison}{\color{blue!70}{[link]}}}{}{
\begin{itemize}
\item Developed an overall ranking heuristic for incomplete pairwise preferences using numeric video metadata.
\item Built an algorithm using Support Vector Regressor and word vector representations to score videos based on textual metadata.
\end{itemize}
}

\cventry{Mar,2015}{Intelligent Food Resources Monitoring \& Management system (PingMyFood - The Food Network)}{Startup venture}{\textsc{SSN Entrepreneurship Development Cell}}{\href{https://github.com/tpsatish95/PingMyFood}{\color{blue!70}{[link]}}}{
\begin{itemize}
\item Built a social network for food, that collaboratively mitigates food resource wastage by routing the surplus food to food deficit regions and also allows anyone to share food with each other.
\item Coupled with intelligent cuisine and chef based topic modeling and quality rating systems.
\end{itemize}
}

\cventry{Sep,2014 - Jun,2015}{Software Controlled Appliances for Energy Conservation and Differently Abled people}{Funded research project}{\textsc{SSN Innovation Center}}{\href{https://github.com/tpsatish95/home-automation-system}{\color{blue!70}{[link]}}}{
\begin{itemize}
\item Built a Raspberry Pi powered system to provide planet-wide access to electronic appliances using software interfaces (mobile apps) via the internet cloud (Internet of Things).
\end{itemize}
}

\cventry{Sep,2015}{Market Segmentation based on Local Customer Activity}{Business Analytics Hackathon}{\textsc{SSN School of Advanced Career Education}}{\href{https://github.com/tpsatish95/market-segmentation-based-on-human-activity}{\color{blue!70}{[link]}}}{
\begin{itemize}
\item Developed a market segmentation algorithm by clustering human activity patterns tracked using smartphone data such as the accelerometer \& gyroscope readings.
\end{itemize}
}
\cventry{Sep,2014}{Energy Demand Prediction System using Smart Meters}{Research project}{\textsc{SSN Innovation Center}}{}{
\begin{itemize}
\item Researched on developing an Arduino powered smart meter system with real-time data sourcing and mining techniques using distributed computing technologies.
\end{itemize}
}
\cventry{Dec,2013}{Regional Transport Office (RTO) Management System}{Software Development Engineer - internship}{\textsc{Ramco Systems, India}}{\href{https://github.com/tpsatish95/misc-projects-and-experiments/tree/master/redundant_databases_ramco_systems}{\color{blue!70}{[link]}}}{
\begin{itemize}
\item Worked on building and managing fail-safe redundant database systems.
\item Learned about Enterprise Resource Management (ERP) on cloud.
\end{itemize}}

%%%%%%%%%%%%%%%%%%%%%%%%%%%%%%%%%%%%%%%%%%%%%%%%%%%%%%%%%%%%%%%%%%%%%%%%%%%%

\section{Research Papers and Patents}
\cvitem{Feb,2017}{Research paper titled "\textbf{Deep Learning the Indus Script}", arXiv:1702.00523v1 preprint. \href{https://arxiv.org/pdf/1702.00523.pdf}{\color{blue!70}{[link]}}}

\cvitem{Apr,2015}{Paper titled "\textbf{Home Automation Systems - A Study}", International Journal of Computer Applications (IJCA), Vol. 116 - No.11, Reference ID: pxc3902601. \href{http://research.ijcaonline.org/volume116/number11/pxc3902601.pdf}{\color{blue!70}{[link]}} \newline
- \textbf{Best paper award} at the SSN UG paper presentation event and has 7 citations on \href{https://scholar.google.co.in/citations?user=gNr8v84AAAAJ&hl=en&oi=ao}{\color{blue!70}{Google Scholar}}.}

\cvitem{Oct,2015}{Patent filed under the title "\textbf{Universally Compatible and Accessible, Software Controlled, Expandable Home Automation System, for Energy Conservation and the Differently-Abled}", Reference ID: 5729/CHE/2015. \href{https://ipindiaonline.gov.in/patentsearch/search/index.aspx}{[patent search link]}}

\cvitem{Apr,2015}{Paper titled "\textbf{Automated Meter Reading System - A Study}", International Journal of Computer Applications (IJCA), Vol. 116 - No.18, Reference ID: pxc3902783. \href{http://research.ijcaonline.org/volume116/number18/pxc3902783.pdf}{\color{blue!70}{[link]}}}

\cvitem{Dec,2014}{Poster presented, titled "\textbf{Text Based Emotion Recognition System}", CMU-IPTSE Winter School, at NIT Surathkal. \href{https://github.com/tpsatish95/emotion-detection-from-text/blob/master/docs/poster.jpg}{\color{blue!70}{[link]}}}

%%%%%%%%%%%%%%%%%%%%%%%%%%%%%%%%%%%%%%%%%%%%%%%%%%%%%%%%%%%%%%%%%%%%

\section{Open source contributions}
\cventry{\textbf{Diskoveror}}{Text analytics package}{}{}{}{
\begin{itemize}
\item Developed the Topic modeling and Sentiment Analysis modules in Python. \href{https://github.com/serendio-labs-stage/diskoveror-ml-trainer}{\color{blue!70}{[link]}}
\item Interfaced Python and Java code bases with Facebook's Thrift API. \href{https://github.com/serendio-labs-stage/diskoveror-ta/commits?author=tpsatish95}{\color{blue!70}{[link]}}
\end{itemize}
}
\cventry{\textbf{RealImage}}{Algorithm for territorial restriction of film distribution rights}{Challenge 2016}{}{}{
\begin{itemize}
\item Implemented using hash maps, trees and bit indexes with O(1) lookup time complexity. \href{https://github.com/RealImage/challenge2016/pull/6}{\color{blue!70}{[link]}}
\end{itemize}
}
\cventry{\textbf{Keras}}{Deep Learning library for Python}{}{}{}{
\begin{itemize}
\item Implemented random shear image data augmentation to manipulate data while training CNNs. \href{https://github.com/fchollet/keras/commit/5cf50f6a6cb7d0ed5c309dbb7a647ad2ce454322}{\color{blue!70}{[link]}}
\end{itemize}
}
\cventry{\textbf{Gensim}}{Topic modeling library for Python}{}{}{}{
\begin{itemize}
\item Ported the phrase vector representation technique of word2vec word vectors, from C to Python. \href{https://github.com/RaRe-Technologies/gensim/commits?author=tpsatish95}{\color{blue!70}{[link]}}
\end{itemize}
}

%%%%%%%%%%%%%%%%%%%%%%%%%%%%%%%%%%%%%%%%%%%%%%%%%%%%%%%%%%%%%%%%%%%%%%%

\section{Awards and Achievements}

\cvitem{}{- Certified for Proficiency in "Design and Analysis of Algorithms" by \textbf{Microsoft Research}.}
\cvitem{}{- \textbf{$2^{nd}$ Runners-up} in \textbf{Smart India Hackathon} (Dept. of Posts), organized by the Govt. of India.}
\cvitem{}{- \textbf{Top 4} in the state and \textbf{top 100} across the country,  Aspirations 2020 programming contest, \textbf{Infosys}.}
\cvitem{}{- \textbf{Outstanding Student Organizer Award}, SSN ACM Student Chapter.}
\cvitem{}{- \textbf{Merit Scholarship} for the 1st academic year, worth Rs.105,000, SSN CE.}
\cvitem{}{- \textbf{Young Achiever Award} for Excellence in Academics, Tractor And Farm Equipment Ltd., India.}
\cvitem{}{- All India Rank 456 in 14th \textbf{National Science Olympiad}.}
\cvitem{}{- \textbf{Runner-up} at "Codigo", \textbf{competitive programming} contest at "Reboot", a national level technical symposium of the SSN School of Management.}
\cvitem{}{- Rated as an \textbf{excellent programmer in "C"} by NIIT.}

%%%%%%%%%%%%%%%%%%%%%%%%%%%%%%%%%%%%%%%%%%%%%%%%%%%%%%%%%%%%%%%%%%

\section{Technical Skills}
\cvitem{Languages}{Python, C, C++, Java, R, VB.Net}
\cvitem{Libraries}{\textsc{Caffe, Keras, OpenCV}, Scikit-Learn, NLTK}
\cvitem{Others}{Linux, Android SDK, Git, \LaTeX, Docker, Adobe Photoshop}

%%%%%%%%%%%%%%%%%%%%%%%%%%%%%%%%%%%%%%%%%%%%%%%%%%%%%%%%%%%%%%%%%%

\section{Professional Activities}
\cventry{Jun,2014 - May,2016}{Association for Computing Machinery (ACM)}{Student Chapter}{SSN CE}{}{
\begin{itemize}
\item Roles: \textbf{Chairman} (Jul,2015 to May,2016), \textbf{Treasurer and Technology Lead} (Jun,2014 to Jul,2015)
\end{itemize}}

\cventry{Jul,2015 - May,2016}{Indian Society for Technical Education (ISTE)}{Student Chapter}{SSN CE}{}{
\begin{itemize}
\item Role: \textbf{Vice President}
\end{itemize}}

\cventry{Jun,2014 - Jul,2015}{Google Student Club (GSC)}{}{SSN CE}{}{
\begin{itemize}
\item Role: \textbf{Technology Lead}
\end{itemize}}

\cventry{Jun,2015 - May,2016}{Placement Cell}{}{SSN CE}{}{
\begin{itemize}
\item Role: \textbf{Student Placement Coordinator}, Computer Science department
\end{itemize}}

%%%%%%%%%%%%%%%%%%%%%%%%%%%%%%%%%%%%%%%%%%%%%%%%%%%%%%%%%%%%%%%%%%%%%%

\section{Extracurricular Activities}
\cvitem {}{- \textbf{Tech-talk} on "ML from a CV and NLP perspective", Real Image Media Technologies \href{https://github.com/tpsatish95/misc-projects-and-experiments/blob/master/real-image/ML-OffSite-TechTalk.pdf}{\color{blue!70}{[link]}.}}
\cvitem {}{- \textbf{Resource person} for a two-day workshop on "ML using Python", ACM Student Chapter \href{https://github.com/tpsatish95/Python-Workshop}{\color{blue!70}{[link]}.}}
\cvitem {}{- \textbf{Organizing head} of “Open Programming”, on-site competitive programming event in "Paradigm" (National level technical symposium, SSN CE).}
\cvitem {}{- Designed the initial website for SSN ACM Student Chapter (currently archived).}
\cvitem {}{- Member of “\textbf{Teach A School}”, an initiative to provide better education for the underprivileged.}
\cvitem {}{- Won the \textbf{creative ad making} contest (ADZAP) at "Prodigy", state level tech-fest, Anna University.}
\cvitem {}{- Web browser developed in VB.Net displayed at "Prodigy", Anna University \href{https://github.com/tpsatish95/misc-projects-and-experiments/tree/master/web-browser}{\color{blue!70}{[link]}}.}

\section{Massive Open Online Courses}
\centering
\cvitem{\textbf{Stanford}}{
\centering
\begin{tabularx}{450pt}{@{\extracolsep{\fill}}>{\centering\arraybackslash}c c c}
Machine Learning & Natural Language Processing & CNN for Visual Recognition\\
Prof. Andrew Ng & Prof. Daniel Jurafsky & Dr. Andrej Karpathy \\
Coursera & Coursera & CS231n
\end{tabularx}
}

%%%%%%%%%%%%%%%%%%%%%%%%%%%%%%%%%%%%%%%%%%%%%%%%%%%%%%%%%%%%%%%%%%

\clearpage


\end{document}
