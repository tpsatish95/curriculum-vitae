%%%%%%%%%%%%%%%%%%%%%%%%%%%%%%%%%%%%%%%%%%%%%%%%%%%%%%%%%%%%%%%%%%%%%
% Author: Maria Luiza Linhares Dantas
% 
% This is an example of a very complete CV using the 'moderncv' package
% and the 'timeline' package. For more information on those, please
% access: 
% https://www.ctan.org/tex-archive/macros/latex/contrib/moderntimeline
% https://www.ctan.org/tex-archive/macros/latex/contrib/moderncv
%%%%%%%%%%%%%%%%%%%%%%%%%%%%%%%%%%%%%%%%%%%%%%%%%%%%%%%%%%%%%%%%%%%%%
%\title{Curriculum Vitae}
\documentclass[11pt, a4paper, roman]{moderncv}
\moderncvstyle{classic}
\moderncvcolor{blue} 
\pagenumbering{roman}
\usepackage{tabularx}
\usepackage[T1]{fontenc}
\usepackage[utf8]{inputenc}
\usepackage[scale=0.85]{geometry}    % Width of the entire CV
\setlength{\hintscolumnwidth}{2.2cm} % Width of the timeline on your left
\usepackage{mathptmx}
\usepackage{pdfpages/pdfpages}
\usepackage{moderntimeline/moderntimeline}
\usepackage{xpatch/xpatch}
\usepackage{color, graphicx}
\tlmaxdates{2007}{2016}              % Beggining and start of your timeline       

\usepackage[none]{hyphenat}

\usepackage{lastpage}

\rfoot{\addressfont\itshape{Page \thepage\ }}

\usepackage{array}
\newcolumntype{P}[1]{>{\centering\arraybackslash}p{#1}}


\newcommand{\tab}[1]{\hspace{.2667\textwidth}\rlap{#1}}
\newcommand{\itab}[1]{\hspace{0em}\rlap{#1}}

\newcommand{\cvreferencecolumn}[2]{%
  \cvitem[0.8em]{}{%
    \begin{minipage}[t]{\listdoubleitemmaincolumnwidth}#1\end{minipage}%
    \hfill%
    \begin{minipage}[t]{\listdoubleitemmaincolumnwidth}#2\end{minipage}%
    }%
}

\newcommand{\cvreference}[8]{%
    \textbf{#1}\newline% Name
    \ifthenelse{\equal{#2}{}}{}{\addresssymbol~#2\newline}%
    \ifthenelse{\equal{#3}{}}{}{#3\newline}%
    \ifthenelse{\equal{#4}{}}{}{#4\newline}%
    \ifthenelse{\equal{#5}{}}{}{#5\newline}%
    \ifthenelse{\equal{#6}{}}{}{\emailsymbol~\texttt{\href{mailto:#6}{\nolinkurl{#6}}}\newline}%
    \ifthenelse{\equal{#7}{}}{}{\phonesymbol~#7\newline}
    \ifthenelse{\equal{#8}{}}{}{\mobilephonesymbol~#8}}


% Personal Information
\name{Satish Palaniappan}{}
\title{\emph{Curriculum Vitae}}                              
\social[github]{tpsatish95}   % optional, remove / comment the line if not wanted           
\social[linkedin]{satishpalaniappan}  % optional, remove / comment the line if not wanted   
\email{tpsatish95@gmail.com}                % optional, remove / comment the line if not wanted
\phone[mobile]{+91~9488515784}      

\begin{document}
\pagenumbering{arabic}
\makecvtitle

\section{Interests}
\cventry{}{Machine Learning, Deep Learning, Computer Vision, Natural Language Processing, Algorithm Design, Reinforcement Learning}{}{}{}{}

%%%%%%%%%%%%%%%%%%%%%%%%%%%%%%%%%%%%%%%%%%%%%%%%%%%%%%%%%%%%%%%%%%%%%%%%%

\section{Educational Background}
\cventry{2012-2016}{Sri Sivasubramaniya Nadar College of Engineering (SSN CE)}{Anna University, \newline B.E. Computer Science and Engineering, \emph{\textbf{CGPA}: 8.56/10}}{}{}{}
\cventry{2010-2012}{TVS Matriculation Higher Secondary School (TVS M HSS)}{\newline 12th Grade (Higher Secondary Certificate), \emph{\textbf{Score}: 98\%}}{}{}{}

%%%%%%%%%%%%%%%%%%%%%%%%%%%%%%%%%%%%%%%%%%%%%%%%%%%%%%%%%%%%%%%%%%%%%%%%%%%%

\section{Academic Research Experience}
\cventry{Dec,2015 - Present}{Research Assistant}{\textsc{Institute of Mathematical Sciences, Chennai, India}}{}{}{
\textbf{Optical Character Recognition on Indus Scripts} \href{https://github.com/tpsatish95/indus-script-ocr}{{\color{blue!70}{[link]}}}
\newline
under Prof. Ronojoy Adhikari, Department of Physics
\begin{itemize}
\item To recognize Indus script symbols from scans and photographs of ancient Harappan civilization artifacts.
\item Techniques: Convolutional Neural Networks (CNN), GoogLeNet, Selective Search, and Transfer Learning.
\item This work got featured in \href{http://www.thehindu.com/sci-tech/science/chennai-team-taps-ai-to-read-indus-script/article17448690.ece}{{\color{blue!70}{\textit{The Hindu}}}}, \href{https://www.theverge.com/2017/1/25/14371450/indus-valley-civilization-ancient-seals-symbols-language-algorithms-ai\#EQQA6r}{{\color{blue!70}{\textit{The Verge}}}} (small piece), \href{http://timesofindia.indiatimes.com/city/chennai/app-may-help-decipher-indus-valley-symbols/articleshow/57281369.cms}{{\color{blue!70}{\textit{Times of India}}}}, and \href{http://www.sbs.com.au/yourlanguage/tamil/en/content/app-decipher-ancient-symbols?language=en}{{\color{blue!70}{\textit{SBS Radio - Australia}}}}.
\end{itemize}}

\cventry{Dec,2014}{Research Intern}{IPTSE Winter School, \textsc{Carnegie Mellon University}}{at NIT Surathkal}{}{
\textbf{Text Based Emotion Recognition System} \href{https://github.com/tpsatish95/emotion-detection-from-text}{\color{blue!70}{[link]}}
\newline
under Prof. Bhiksha Raj, Department of Computer Science
\begin{itemize}
\item Classify any textual data, using histograms of word2vec word/phrase clusters, into the 7 basic emotions.
\item Techniques: Word2Vec, Latent Dirichlet Allocation (LDA), K-means, Support Vector Machines (SVM), Probabilistic N-Gram models, Multinomial Naive Bayes.
\end{itemize}}

%%%%%%%%%%%%%%%%%%%%%%%%%%%%%%%%%%%%%%%%%%%%%%%%%%%%%%%%%%%%%%%%%%%%%%%%%%%%%%%

\section{Industry Research Experience}
\cventry{Jun,2016 - Present}{Software Engineer}{\textsc{Qube Cinema Technologies, Chennai, India}}{}{}{
\begin{itemize}
\item Projects: 
\begin{itemize}
\item Scalable, \textbf{deep-learned, viewer-demographics mining} (count, age, gender, and emotions of the movie watchers), from low-light images of a theatre's auditorium.
\item Real-time, adaptive, partner \textbf{selection algorithm} for businesses.
\item \textbf{Intelligent bot for syncing theatre databases} around the globe into a unified format.
\item \textbf{Video super resolution} of a movie-clipping by up to four times, with minimal loss in clarity.
\end{itemize}
\item Techniques and Tools: CNN with master-child model and human-feedback-based learning, Knapsack Problem, Minimum Cost Flow Problem (Transportation Problem), Pruned Search Trees, Linked and Recursive Interval Trees, Generative Adversarial Networks (GAN), Camera Calibration, Amazon Web Services, and Flask+uWSGI+Nginx servers.
\end{itemize}}

\cventry{May,2015 - Jul,2015}{Data Scientist - Intern}{\textsc{Serendio Inc., Chennai, India}}{}{}{
\begin{itemize}
\item Projects:
\begin{itemize}
\item Universal multi-domain \textbf{sentiment scorer} for text  \href{https://github.com/tpsatish95/Universal-MultiDomain-Sentiment-Classifier}{\color{blue!70}{[link]}}.
\item \textbf{Topic composition modeling} using hierarchical K-Means and semantic word clusters \href{https://github.com/tpsatish95/Topic-Modeling-Social-Network-Text-Data}{\color{blue!70}{[link]}}.
\item \textbf{Internet slang text parser} \href{https://github.com/tpsatish95/SocialTextFilter}{\color{blue!70}{[link]}}.
\end{itemize}
\item Techniques and Tools:  Gensim, Tf-idf features, Bagging and Boosting, CMU ARK's Twokenize, Rake - Keyword Extractor, Web Crawlers, etc.
\item \textbf{Open Source Community Manager - Intern}, at Serendio Labs, the open source division of Serendio Inc.
\item Serendio Inc.'s \textbf{campus ambassador} at SSN CE.
\end{itemize}
}

%%%%%%%%%%%%%%%%%%%%%%%%%%%%%%%%%%%%%%%%%%%%%%%%%%%%%%%%%%%%%%%%%%%%%%%%%%%%%%%%

\section{Projects}
\cventry{Jun,2015 - Apr,2016}{Automated Scenario Description for Images}{Undergraduate thesis}{\textsc{Anna University}}{}{under Prof. Milton R.S., Department of Computer Science, SSN CE, \href{https://github.com/tpsatish95/image-captioning}{\color{blue!70}{[link]}}.
\begin{itemize}
\item Harnessed object identification and scene classification techniques to automatically generate natural language descriptions of images. 
\end{itemize}}

\cventry{Feb,2015 - Dec,2015}{Automated Scoring of YouTube videos from Pairwise Comparisons and Metadata based on Degree of Funniness}{Side project}{\href{https://github.com/tpsatish95/Youtube-Comedy-Comparison}{\color{blue!70}{[link]}}}{}{   
\begin{itemize}
\item Developed a ranking heuristic based on incomplete pairwise user preferences, using numeric video metadata.
\item Built a model using Support Vector Regressor and word vector representations to score videos based on textual metadata.
\end{itemize}
}

\cventry{Mar,2015}{Intelligent Food Resources Monitoring \& Management system (PingMyFood - The Food Network)}{Startup venture}{\textsc{SSN Entrepreneurship Development Cell}}{\href{https://github.com/tpsatish95/PingMyFood}{\color{blue!70}{[link]}}}{ 
\begin{itemize}
\item Built a social network for sharing food, that collaboratively mitigates food resource wastage by routing the surplus food to food deficit regions and also allows anyone to share food with each other.
\item Coupled with an intelligent, cuisine and chef based, topic modeling and quality rating system.
\end{itemize}
}

\cventry{Sep,2014 - Jun,2015}{Software Controlled Appliances for Energy Conservation and Differently Abled people}{Funded research project}{\textsc{SSN Innovation Center}}{\href{https://github.com/tpsatish95/home-automation-system}{\color{blue!70}{[link]}}}{ 
\begin{itemize}
\item Built a Raspberry Pi powered system to provide planet-wide access to electronic appliances using software interfaces (mobile apps) via the internet cloud (Internet of Things).
\end{itemize}
}

\cventry{Sep,2015}{Market Segmentation based on Local Customer Activity}{Business Analytics Hackathon}{\textsc{SSN School of Advanced Career Education}}{\href{https://github.com/tpsatish95/market-segmentation-based-on-human-activity}{\color{blue!70}{[link]}}}{ 
\begin{itemize}
\item Developed a market segmentation algorithm by clustering human activity patterns tracked using smartphone data such as the accelerometer \& gyroscope readings.
\end{itemize}
}
\cventry{Dec,2013}{Regional Transport Office (RTO) Management System}{Software Development Engineer - internship}{\textsc{Ramco Systems, India}}{\href{https://github.com/tpsatish95/misc-projects-and-experiments/tree/master/redundant_databases_ramco_systems}{\color{blue!70}{[link]}}}{
\begin{itemize}
\item Worked on building and managing fail-safe redundant database systems.
\item Learned about Enterprise Resource Management (ERP) on cloud.
\end{itemize}}

%%%%%%%%%%%%%%%%%%%%%%%%%%%%%%%%%%%%%%%%%%%%%%%%%%%%%%%%%%%%%%%%%%%%%%%%%%%%

\section{Research Papers and Patents}
\cvitem{Feb,2017}{Research paper titled "\textbf{Deep Learning the Indus Script}", arXiv:1702.00523v1 preprint. \href{https://arxiv.org/pdf/1702.00523.pdf}{\color{blue!70}{[link]}}}

\cvitem{Apr,2015}{Paper titled "\textbf{Home Automation Systems - A Study}", International Journal of Computer Applications (IJCA), Vol. 116 - No.11, Reference ID: pxc3902601. \href{http://research.ijcaonline.org/volume116/number11/pxc3902601.pdf}{\color{blue!70}{[link]}} \newline
- \textbf{Best paper award} at the SSN UG paper presentation event and has 12 citations \href{https://scholar.google.co.in/citations?user=gNr8v84AAAAJ&hl=en&oi=ao}{\color{blue!70}{[Google Scholar]}}.}

\cvitem{Oct,2015}{Patent filed under the title "\textbf{Universally Compatible and Accessible, Software Controlled, Expandable Home Automation System, for Energy Conservation and the Differently-Abled}", Reference ID: 5729/CHE/2015. \href{http://ipindiaservices.gov.in/PublicSearch/}{[patent search link]}}

\cvitem{Apr,2015}{Paper titled "\textbf{Automated Meter Reading System - A Study}", International Journal of Computer Applications (IJCA), Vol. 116 - No.18, Reference ID: pxc3902783. \href{http://research.ijcaonline.org/volume116/number18/pxc3902783.pdf}{\color{blue!70}{[link]}}}

\cvitem{Dec,2014}{Poster presented, titled "\textbf{Text Based Emotion Recognition System}", CMU-IPTSE Winter School, at NIT Surathkal. \href{https://github.com/tpsatish95/emotion-detection-from-text/blob/master/docs/poster.jpg}{\color{blue!70}{[link]}}}

%%%%%%%%%%%%%%%%%%%%%%%%%%%%%%%%%%%%%%%%%%%%%%%%%%%%%%%%%%%%%%%%%%%%%%

\section{Talks}
\cventry{}{- Deep learning based OCR engine for the Indus script}{}{}{}{
\begin{itemize}
\item Venues: \textbf{Indian Deep Learning Initiative} (IDLI) \href{https://github.com/tpsatish95/talks/blob/master/Deep\%20learning\%20based\%20OCR\%20engine\%20for\%20the\%20Indus\%20script\%20-\%20IDLI\%20Talk.pdf}{\color{blue!70}{[slide deck]}} \href{https://www.youtube.com/watch?v=qPF1oR9yMNY}{\color{blue!70}{[video]}} \href{https://www.facebook.com/groups/idliai/}{\color{blue!70}{[link]}}, \textbf{ThoughtWorks Geek Night} \href{https://github.com/tpsatish95/talks/blob/master/Deep\%20learning\%20based\%20OCR\%20engine\%20for\%20the\%20Indus\%20script\%20-\%20TW\%20Geek\%20Night.pdf}{\color{blue!70}{[slide deck]}} \href{https://www.youtube.com/watch?v=g7v4QaCD-UQ}{\color{blue!70}{[video]}} \href{https://twchennai.github.io/geeknight/edition-43.html}{\color{blue!70}{[link]}}, \textbf{ChennaiPy} \href{http://chennaipy.org/may-2017-meet-minutes.html}{\color{blue!70}{[link]}}, \textbf{Anthill Inside 2017} \href{https://anthillinside.talkfunnel.com/2017/15-deep-learning-based-ocr-engine-for-the-indus-scrip}{\color{blue!70}{[proposal]}}.
\end{itemize}}

\cventry{}{- Pokemon World and Indus Valley Civilisation - The Analogy}{at Qube Cinema Technologies - Offsite 2017 {\color{blue!70}{slide deck}} \href{https://docs.google.com/presentation/d/1C2peTKpVMQM1KuadNgPzR7dg_R7cqexP0B8VdifsfX4/edit?usp=sharing}{\color{blue!70}{[V1}}, \href{https://docs.google.com/presentation/d/1FSodM6Qbzglhrn_VizEul-MkerEBhEQIHpXIgPZGlG4/edit?usp=sharing}{\color{blue!70}{V2]}}}{}{}{}

\cventry{}{- ML from a CV and NLP perspective}{at Qube Cinema Technologies - Offsite 2016 \href{https://github.com/tpsatish95/talks/blob/master/ML\%20from\%20a\%20CV\%20and\%20NLP\%20perspective.pdf}{\color{blue!70}{[slide deck]}}}{}{}{}

\cventry{}{- Python Hands-on}{Two-day workshop, at ACM Student Chapter, SSN CE \href{https://github.com/tpsatish95/Python-Workshop}{\color{blue!70}{[link]}}}{}{}{}

%%%%%%%%%%%%%%%%%%%%%%%%%%%%%%%%%%%%%%%%%%%%%%%%%%%%%%%%%%%%%%%%%%%%

\section{Open source contributions}
\cventry{\textbf{Diskoveror}}{Text analytics package}{}{}{}{ 
\begin{itemize}
\item Developed the Topic modeling and Sentiment Analysis modules in Python. \href{https://github.com/serendio-labs-stage/diskoveror-ml-trainer}{\color{blue!70}{[link]}}
\item Interfaced Python and Java code bases with Facebook's Thrift API. \href{https://github.com/serendio-labs-stage/diskoveror-ta/commits?author=tpsatish95}{\color{blue!70}{[link]}}
\end{itemize}
}
\cventry{\textbf{RealImage}}{Algorithm for territorial restriction of film distribution rights}{Challenge 2016}{}{}{ 
\begin{itemize}
\item Implemented using hash maps, trees and bit indexes with O(1) lookup time complexity. \href{https://github.com/RealImage/challenge2016/pull/6}{\color{blue!70}{[link]}}
\end{itemize}
}
\cventry{\textbf{Keras}}{Deep Learning library for Python}{}{}{}{ 
\begin{itemize}
\item Implemented random shear image data augmentation to manipulate data while training CNNs. \href{https://github.com/fchollet/keras/commit/5cf50f6a6cb7d0ed5c309dbb7a647ad2ce454322}{\color{blue!70}{[link]}}
\end{itemize}
}
\cventry{\textbf{Gensim}}{Topic modeling library for Python}{}{}{}{ 
\begin{itemize}
\item Ported the phrase vector representation technique of word2vec word vectors, from C to Python. \href{https://github.com/RaRe-Technologies/gensim/commits?author=tpsatish95}{\color{blue!70}{[link]}}
\end{itemize}
}

%%%%%%%%%%%%%%%%%%%%%%%%%%%%%%%%%%%%%%%%%%%%%%%%%%%%%%%%%%%%%%%%%%%%%%%

\section{Awards and Achievements}

\cvitem{}{- \textbf{Mentored and advised a tech startup} that digitizes hardware inventory management in post offices and makes it more intelligent, we were the \textbf{$2^{nd}$ Runners-up} in \textbf{Smart India Hackathon} (Dept. of Posts), organized and \textbf{incubated by the Government of India}.}
\cvitem{}{- \textbf{Microsoft Research} certified, for proficiency in "Design and Analysis of Algorithms".}
\cvitem{}{- \textbf{Top 4} in the state and \textbf{top 100} across the country,  Aspirations 2020 programming contest, \textbf{Infosys}.}
\cvitem{}{- \textbf{Merit Scholarship} for the 1st academic year, worth Rs.105,000, SSN CE.}
\cvitem{}{- \textbf{Young Achiever Award} for Excellence in Academics, Tractor And Farm Equipment Ltd., India.}
\cvitem{}{- \textbf{Outstanding Student Organizer Award}, SSN ACM Student Chapter.}
\cvitem{}{- All India Rank 456 in 14th \textbf{National Science Olympiad}.}
\cvitem{}{- Rated as an \textbf{excellent programmer in "C"} by NIIT.}

%%%%%%%%%%%%%%%%%%%%%%%%%%%%%%%%%%%%%%%%%%%%%%%%%%%%%%%%%%%%%%%%%%

\section{Technical Skills}
\cvitem{Languages}{Python, C, C++, Java, R, VB.Net}
\cvitem{Libraries}{\textsc{Caffe, TensorFlow, Keras, OpenCV}, Scikit-Learn, NLTK}
\cvitem{Others}{Linux, Android SDK, Git, \LaTeX, Docker, Adobe Photoshop}

%%%%%%%%%%%%%%%%%%%%%%%%%%%%%%%%%%%%%%%%%%%%%%%%%%%%%%%%%%%%%%%%%%

\section{Professional Activities}
\cventry{2014 - 2016}{Association for Computing Machinery (ACM)}{Student Chapter}{SSN CE}{}{
\begin{itemize}
\item Roles: \textbf{Chairman} (2015 to 2016), \textbf{Treasurer and Tech Lead} (2014 to 2015)
\end{itemize}}

\cventry{2015 - 2016}{Indian Society for Technical Education (ISTE)}{Vice President, Student Chapter}{SSN CE}{}{}
\cventry{2014 - 2015}{Google Student Club (GSC)}{Tech Lead}{SSN CE}{}{}
\cventry{2015 - 2016}{Placement Cell}{Student Placement Coordinator, Computer Science Department}{SSN CE}{}{}

%%%%%%%%%%%%%%%%%%%%%%%%%%%%%%%%%%%%%%%%%%%%%%%%%%%%%%%%%%%%%%%%%%

\section{Massive Open Online Courses}
\centering
\cvitem{\textbf{Stanford}}{
\centering
\begin{tabularx}{450pt}{@{\extracolsep{\fill}}>{\centering\arraybackslash}c c c} 
Machine Learning & Natural Language Processing & CNN for Visual Recognition\\
Prof. Andrew Ng & Prof. Daniel Jurafsky & Dr. Andrej Karpathy \\
Coursera & Coursera & CS231n
\end{tabularx}
}

%%%%%%%%%%%%%%%%%%%%%%%%%%%%%%%%%%%%%%%%%%%%%%%%%%%%%%%%%%%%%%%%%%

\clearpage


\end{document}
